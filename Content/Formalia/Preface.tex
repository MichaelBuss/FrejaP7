\chapter*{Preface}
\lipsum[1]

%Her er vores gamle preface til inspiration

%The project is written by 4$^{th}$ semester nanotechnology students of group 5.342 from Aalborg University. The project was written in the period February $4^{th}$ to May $23^{rd}$ under supervision from Peter Fojan. The project is focused on the topic of synthesising and studying stimuli responsive peptide hydrogels. Two almost identical peptides sequences, NH$_2 - $KVFSWSFVD$ - $COOH (P9(S)) and  NH$_2 - $KVFDWDFVD$ - $COOH (P9(D)), were designed and synthesised. The peptides were analysed and purified, followed by critical gelation concentration determination, alongside temperature- and pH-dependency tests using a variety of methods such as high performance liquid chromatography, absorption spectroscopy, circular dichroism spectroscopy, fluorescence spectroscopy and atomic force microscopy. The project will cover the theory behind peptide hydrogels, their primary structure design, and how they respond to different environmental conditions. In addition, the theory behind the utilised measuring and analysing methods described above will be covered. This is followed by a method chapter where the synthesis, purification, and analysis of the peptides are described further. The results section will include the experimentally obtained data, that will later be interpreted, compared, and discussed in the chapter 'Discussion'. The findings will finally be summarised in the chapter 'Conclusion'. The report is written in British English and will be using the Oxford comma. The Vancouver reference system is used, meaning citations will be given a number that corresponds to a source in the bibliography. Figures without citations were made by this group.  


%Her er vores gamle preface til inspiration

%The project is written by group B131 which consists of second semester nanotechnology students at Aalborg University from February $1^{st}$ to May $29^{th}$ with lector Peter Fojan as supervisor and Carla Smink as co-supervisor. The topic of the project is fabrication of electrodes for utilization in dye-sensitised solar cells via electrospinning. This project will cover the method of electrospinning, dye-sensitised solar cells, annealing, as well as the measuring methods Fourier transformation Infrared Spectroscopy and X-ray Diffraction. It also contains a short introduction, "state of the art", which briefly covers the issues the environment faces due to our utilization of resources, the steps that have been taken to combat these, and how nanotechnology is useful in this endeavour. In addition, it briefly covers solar cells and the method of electrospinning. This project includes theoretical chapters which describe the electrospinning process, dye-sensitised solar cells, annealing and the measuring methods. This is followed by a method chapter which describes the fabrication methods, the measuring methods as well as a third measuring method, Atomic Force Microscopy, as they were utilized in the experiments. This leads into a chapter describing the results of the experiments, followed by a discussion and conclusion. This project uses oxford comma and British English. Each chapter, section, subsection, figure, table and equation is numbered. The reference system used is Vancouver, which represents every citation as a [number], that refers directly to a specific source in the bibliography. Figures with no references were created by the group itself.

\blankline \begin{center}
    Aalborg University, December xx, 2021
\end{center}








