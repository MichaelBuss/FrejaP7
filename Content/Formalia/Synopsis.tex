
\lipsum[1]

%Det her er vores abstract fra sidste semester, i kan bruge det som inspiration hvis i har lyst 

%The aim of this project is to design, synthesise, and subsequently analyse pH sensitive peptides self-assembling into hydrogels. The peptides P9(S) and P9(D) were synthesised via the Fmoc solid phase peptide synthesis method, followed by high performance liquid chromatography analysis of both peptides, and purification of P9(S). Furthermore both peptides were determined to form $\beta -$sheet structures as well as random coil, via circular dichroism spectroscopy. It was determined that P9(S) has an optimal hydrogelation at pH 7.0, with a concentration above 150 $\mu$M and P9(D) at pH 3.5, with a concentration of 100 $\mu$M. Both peptides were determined to be pH responsive and temperature stable, with P9(D) having a more abrupt pH response, and appearing more temperature stable. It was found, via atomic force microscopy measurements, that P9(S) formed fibrillar network at a pH near 4.65 as well as sphere structures. P9(D) were found to form globular structures at pH 3.56.


%Det her er vores abstract fra sidste semester, i kan bruge det som inspiration hvis i har lyst 

%The aim of this \semester \phantom{,,}semester project is to create aligned zinc oxide nanofibres utilizing electrospinning and study the effect the content of the precursor salt had on the morphology of the fibres, and how the annealing process changes the composition of the fibres. This was achieved by electrospinning five samples with different salt concentrations over a set of parallel electrodes while moving the spinning head. Then the morphology of the fibres was measured with Atomic Force Microscopy, it was found that the fibres made with a low concentration of precursor salt became disjointed as the polymer was removed, while the fibres made with a high concentration of salt remained continuous through the annealing process. Likewise, it was shown that at lower concentrations of salt, the reduction of height was greater than at higher concentrations of salt. By doing Fourier transformation infrared spectroscopy and X-ray diffraction measurements on a sample done with 12\% salt, it was shown that during annealing, the polymer and precursor salt were removed and zinc oxide was created. However, post-annealing, the material disintegrated into powder, and could therefore not readily be used as a photoanode in a dye-sensitised solar cell. 


