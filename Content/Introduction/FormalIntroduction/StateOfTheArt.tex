%\chapter{Introduction and Theoretical Background}
\section{State of the Art}

% Theory outline
% State of the art/introduction
% systemic drug delivery is not great, targeted drug delivery, bio-compatibility, monodispersed droplets, microfluidics is a good candidates 
% nano carriers and drug release (also loading)

%Nanomedicince 

Microfluidics has gained popularity among researchers in a wide range of fields including biotechnology, biological analysis, medicine, and nanotechnology, to mention a few \parencite{carla2019,Chen2013,Rabiee2021,Scheler2019}. The reason behind its increasing demand in the various fields is its promise to reduce costs by minimizing reagent usage and waste \parencite{Pinto2014}. Typical laboratory operations can be performed in microfluidic systems with a fraction of the volume of reagents in significantly less time. This is the basis of ‘lab-on-a-chip’ applications. By making use of microfluidics, micro and nano-sized particles and morphology can be precisely tuned \parencite{Pinto2014}.
\blankline

In particular, the advances in the integration of tools to create, analyse, and manipulate droplets have made microfluidic devices increasingly flexible \parencite{Nge2013}.

Platforms based on microfabrication and microfluidics have been successfully utilised in a variety of applications such as drug discovery, polymer chain reaction (PCR), single-celled analysis, or crystallisation proteins, to name a few. For all these applications, producing droplets of controlled sizes as well as controlled generation rates appear as key factors.\parencite{Krsti2018,Li2018}

Microparticles fabricated by droplet microfluidics have great potential in the fields of drug delivery, cell biology, and biosensors. Materials with good biocompatibility and controlled biodegradability are required to form such microparticles. Microparticles with sizes ranging from a few microns to hundreds of microns can be generated by droplet microfluidics. They are ideal vehicles for drug delivery through oral or subcutaneous administration (insulin is an example of this type of injection as it requires continuous delivery but at a low dose rate). However, the ideal size of the drug-loaded particles should be at sub-micron scales to avoid occlusion of blood vessels.

Due to the various applications of microparticles in a wide range of biomedical applications, such as drug delivery, tissue engineering, biosensing, and cellular life science, as mentioned above, their applications depend on their properties, which correlate with their size, structure, composition, and configuration. Therefore, it is essential to fabricate microparticles in a controlled manner to improve their pharmaceutical capability and reliability for biological studies. However, it has long been a challenge to produce microparticles with such desired properties through conventional methods, including emulsion polymerization, dispersion polymerization, and spray drying. These methods normally result in microparticles with large polydispersity, poor reproducibility, limited functionality, and less tuneable morphology. Thus, development of efficient methods for producing particles with nanoscale size is crucial for effective drug delivery \parencite{Liu2014,Patarroyo2020}.

% For instance, as drug delivery vehicles microcapsules or multi-core microparticles can be prepared with well-defined structures and compositions that allow for high encapsulation efficiency and well-controlled release of the encapsulants. As cell carrier’s (ECM) to protect cells from the surrounding environment and maintain efficient nutrient and metabolic exchanges for long term cell culture. As a result, these cell-laden microparticles have direct applications in tissue engineering stem cell therapy and single cell studies.\parencite{Saldin2017,PrezLuna2018}
% \blankline
%-----------------------------------------------
One of the innovative approach to solve the continual demands of inefficient drug delivery strategies is the usage of nanoparticles (NPs) \parencite{Park2013}. Particles with size in the range 10-1000 nm are known as nanoparticles \parencite{strambeanu2014}. 
NPs have various advantages due to their flexibility of production of desired material and usage of materials that can increase solubility of encapsulated cargo while decreasing toxicity by allowing a controlled release and tissue-specific delivery.\parencite{samir2016,scott2009}

The advantages of NPs are numerous in the field of nanomedicine, where some examples of such applications are targeted drug delivery and development of new medicines that are safer \parencite{nitinreview}.

Various factors such as the required nanoparticle size, toxicity, biodegradability, and biocompatibility should be taken into consideration in selection of matrix material to be used in preparation.

NPs that can deliver therapeutics such as messenger RNA (mRNA) vaccines have been investigated for their high potency and potential for rapid development \parencite{norbert2018,andreas2016}, two mRNA vaccines (mRNA-1273 and BNT162b2, developed by Moderna, Inc. and BioNTech/Pfizer, respectively) for the SARS-CoV-2 virus have gained emergency use authorisation by the U.S. Food and Drug Administration (FDA) and other international agencies \parencite{mrna_1_2020,mrna_2_2021,mrna_3_2021,mrna_4_2020}.\\
Despite the rapid progress in the field of nanomedicine there are still challenges that need to be met. One example of these limitations is the inefficient delivery of NPs to target cells and tissues, which prevents treatments from reaching the performance necessary for clinical use \parencite{dahlman2017,mitchell2020}. 

In order to address the current challenges in NP formulation, microfluidic technologies have been applied to synthesise NPs with more controlled physical properties \parencite{xin2019}.
\\
%This part should be moved to hydrogel section
Here, hydrogel NPs can play an important role in delivery of drugs due to their tunability, high hydrophilicity, high capacity for drug loading, flexibility in size, as well as controlled drug release and smart drug delivery, and remarkable biocompatibility are among the main advantages of this system in drug delivery. 
\\
%Hydrogel NPs as drug delivery carriers are used to avoid the unwanted side effects of the various drugs by their preferential targeted delivery to the tissues of interest and/or changing the time profile of the body exposure to the drug.
%Among the various types of drug delivery systems, hydrogels are considered to be the simplest smart carriers.\parencite{lima2011,}
Hydrogel consists of three-dimensional polymer networks that swell but do not dissolve in an aqueous environment, can be compartmentalised within droplets using microfluidic systems. In particular, hydrogels allow for embedding cells in an aqueous environment that is soft, biocompatible, and stress-protective. Several methods and materials are being tested for cell encapsulation in microbeads whereas the most common hydrogel materials are carbohydrates (such as alginate, agarose, carrageenan, chitosan, gellan gum, and hyaluronic acid (HA)) and proteins (such as collagen, gelatine, fibrin, elastin, and silk fibroin). Each of these methods and materials has been the subject of experimental designs for the encapsulation of a variety of cells. Certain combinations of methods and materials affect the efficiency of cell encapsulation. Efficiency is characterized by cell viability, cell function, microbead size uniformity, and microbead shape. Moreover, the gelation process of the droplets is an important factor in increasing the encapsulation efficiency. The term ‘microbead’ refers to the hydrogel after gelation. Preparation of hydrogel droplets with high monodispersity plays an essential role for droplet microfluidic-based quantitative assays, cell growth under uniform conditions, and controlled material manufacture that depends on the uniformity of the size of the particles that are the building blocks of the hydroge. \parencite{2020,Rashid2019,Nicodemus2008,Vasile2020,Shintaku2006,}
\\
\\
The aim of this project is to understand how hydrogel NPs can be fabricated by employing different methods to fabricate them using a simple T-junction device. In the initial phase the mechanical method is applied, followed by usage of syringe pumps, and in the last step microfluidic devices are used to generate the hydrogel particles.The next step is to understand and test their drug delivery capabilities. The droplet generator device(s) applied will be simulated using COMSOL Multiphysics, where the results will be compared to the experimental achieved.(???)