%This section contain all the exessive info from DropletGeneration section
%----------------------------------------------------------------------------
Microfluidics, the control and study of fluids in the micrometre scale, is gaining popularity among researchers in a wide range of fields including biotechnology, biology analysis, medicine, and nanotechnology, to mention a few. \parencite{Chen2013,Rabiee2021,Scheler2019} The reason behind its increasing demand in the various fields is its promise to reduce costs by minimizing reagent usage and waste \parencite{Pinto2014}, and for mimicking organ behaviour and blood flow in microscale body-on-a-chip \parencite{Selimovi2013}. Typical laboratory operations can be performed in microfluidic systems with a fraction of the volume of reagents in significantly less time. This is the basis of ‘lab-on-a-chip’ applications. Through microfluidics, microbead size and morphology can be precisely tuned.
\blankline

In particular, the advances in the integration of tools to create, analyse, and manipulate droplets have made microfluidic devices increasingly flexible.\parencite{Nge2013}

Platforms based on microfabrication and microfluidics have been successfully utilised in a variety of applications such as drug discovery, polymer chain reaction (PCR), single-celled analysis or crystallisation proteins to name a few. For all these applications, producing droplets at controlled sizes as well as controlled generation rates appear as key factors.\parencite{Krsti2018,Li2018}

Microparticles fabricated by droplet microfluidics have great potential in the fields of drug delivery, cell biology and biosensors. Materials with good biocompatibility and controlled biodegradability are required to form such microparticles. Microparticles with sizes ranging from a few microns to hundreds of microns can be generated by droplet microfluidics. They are ideal vehicles for drug delivery through oral or subcutaneous administration. However, the ideal size of the drug-loaded particles should be at sub-micron scales to avoid occlusion of blood vessels.

Due to the various applications of microparticles in a wide range of biomedical applications, such as drug delivery, tissue engineering, biosensing, and cellular life science as mentioned above their applications depend on their properties which correlate with their size, structure, composition and configuration. Therefore, it is essential to fabricate microparticles in a controlled manner to improve their pharmaceutical capability and reliability for biological studies. However, it has long been a challenge to produce microparticles with such desired properties through conventional methods including emulsion polymerization, dispersion polymerization and spray drying. These methods normally result in microparticles with large polydispersity, poor reproducibility, limited functionality, and less tuneable morphology. Although nanoscale particles can be fabricated by droplet microfluidics low concentration of precursor solutions are used, which reduces the production rate. Thus, development of efficient methods for producing particles with nanoscale size is crucial for effective drug delivery \parencite{Liu2014,Patarroyo2020}.

For instance, as drug delivery vehicles microcapsules or multi-core microparticles can be prepared with well-defined structures and compositions that allow for high encapsulation efficiency and well-controlled release of the encapsulants. As cell carrier’s 
(ECM) to protect cells from the surrounding environment and maintain efficient nutrient and metabolic exchanges for long term cell culture. As a result, these cell-laden microparticles have direct applications in tissue engineering stem cell therapy and single cell studies.\parencite{Saldin2017,PrezLuna2018}
\blankline


%This part should be moved to hydrogel section
Hydrogels, which consists of three-dimensional polymer networks that swell but do not dissolve in an aqueous environment, can be compartmentalised within droplets using microfluidic systems. In particular, hydrogels allow for embedding cells in an aqueous environment that is soft, biocompatible, and stress-protective. Several methods and materials are being tested for cell encapsulation in microbeads whereas the most common hydrogel materials are carbohydrates (such as alginate, agarose, carrageenan, chitosan, gellan gum, and hyaluronic acid (HA)) and proteins (such as collagen, gelatine, fibrin, elastin, and silk fibroin). Each of these methods and materials has been the subject of experimental designs for the encapsulation of a variety of cells. Certain combinations of methods and materials affect the efficiency of cell encapsulation. Efficiency is characterized by cell viability, cell function, microbead size uniformity, and microbead shape. Moreover, the gelation process of the droplets is an important factor in increasing the encapsulation efficiency. The term ‘microbead’ refers to the hydrogel after gelation. Preparation of hydrogel droplets with high monodispersity plays an essential role for droplet microfluidic-based quantitative assays, cell growth under uniform conditions, and controlled material manufacture that depends on the uniformity of size of the hydrogel building block particles. \parencite{2020,Rashid2019,Nicodemus2008,Vasile2020,Shintaku2006}