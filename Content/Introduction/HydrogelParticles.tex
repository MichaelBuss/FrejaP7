\section{Protein Based Hydrogel Particles}
Hydrogels consist of a hydrophillic polymeric network able to absorb wast amounts of water, thus swelling instead of dissolving. The polymer source for the three dimensional network can be either synthetic or natural depending on the desired properties of the hydrogel. Advantages of utilising natural polymers, such as proteins, as the polymer source are their biocompatibility and biodegradability, making them ideal for applications concerning the human body such as tissue engineering and drug delivery.\parencite{Huerta-LópezCarla2021ProteinHydroTheSwiss, PanahiReza2019ProteinHydrogels} 
\par
%
The polymers in the protein hydrogels can be either physically or chemically crosslinked. Physically crosslinked hydrogels are are bound together by non-covalent bonds such as hydrogen bonds and are thus reversible as the crosslinks can be easily broken with a change of the surrounding conditions e.g. temperature. Chemically crosslinked hydrogels on the other hand are held together by covalent bonds making them more stable. \parencite{Huerta-LópezCarla2021ProteinHydroTheSwiss, PanahiReza2019ProteinHydrogels}
%
\subsection{Rheological Properties of Protein Hydrogels}
Hydrogels, and thus protein hydrogels, can be characterised via rheological studies, and through these measurements knowledge about the stiffness and liquid-like behaviour of a hydrogel can be obtained along with information on the strength of the gel network. These viscoelastic properties, i.e. the complex modulus ($G^*$), of a protein hydrogel can be divided up into two terms: the storage modulus ($G'$) and the loss modulus ($G''$). The elasticity of the material is described by the prior, hence the name, as energy can be stored in a material due to its elastic properties. The latter describes the viscous behaviour of the material, again leading to the name, as this represents the loss of energy. These moduli can be related as depicted in \autoref{fig:complexStorageLossMoudlus}, resulting in the expression $\tan(\delta) = G''/G'$. \parencite{PanahiReza2019ProteinHydrogels}
\par
%
\begin{figure}[tbp]
    \centering
    \includegraphics[width=6cm]{example-image}
    \caption{Caption}
    \label{fig:complexStorageLossMoudlus}
\end{figure}
%
In an oscillation rheology measurement $G'$ and $G''$ can be determined through the equations,
\par
%
\begin{align*}
    G' &= \frac{\tau'_{0}}{\gamma},\\
    G'' &= \frac{\tau''_{0}}{\gamma},
\end{align*}
\par
%
\noindent here $\gamma$ is the strain, $\tau$ the stress, and $\tau'_{0}$ and $\tau''_{0}$ the maximum component of the elastic and viscous component of the stress respectively. Moreover the stress and strain can be found from the experiments via the following equations,
\par
%
\begin{align*}
    \tau(t) &= \tau_0 sin(\omega t + \delta),\\
    \gamma(t) &= \gamma_0 sin(\omega t) ,
\end{align*}
\par
%
\noindent where $\delta$ is the phase difference between $\tau$ and $\gamma$, $\omega$ the angular frequency, $\tau_0$ the maximum component of the stress, and $\gamma_0$ the amplitude of oscillation.
For an elastic solid the phase shift between $\gamma$ and $\tau$ is $0\si{\degree}$, as $G'$ is the dominant modulus, and oppositely for a viscous fluid $\tau$ is $90\si{\degree}$ as $G''$ dominates. This can easily be imagined from \autoref{fig:complexStorageLossMoudlus}. However, in the case of a viscoelastic material, such as a protein hydrogel, there are both elastic and viscous properties, thus $\delta$ will have a value between $0\si{\degree}$ and $90\si{\degree}$. The different cases are illustrated in \autoref{fig:phaseShiftMaterials}. \parencite{PanahiReza2019ProteinHydrogels}
\par
%
\begin{figure}[tbp]
    \centering
    \includegraphics[width=6cm]{example-image}
    \caption{Caption}
    \label{fig:phaseShiftMaterials}
\end{figure}
%
To obtain knowledge about the established terms and thereby characterise the protein hydrogel different types of rheological measurements can be made. An example is frequency sweep measurements in which the moduli are determined as a function of frequency. Frequency sweep tests can be used to determine if there have indeed been formed a hydrogel network, as cross-linked hydrogels show solid-like properties, i.e. a $G'$ much larger than the $G''$, at all frequencies. For polymeric melts and polymer solutions which are not cross-linked, $G'$ is larger than $G''$ at high frequencies and $G''$ is larger than $G'$ at low frequencies. This happens as the materials act more like elastic solids and viscous liquids at the respective conditions, thus different from cross-linked hydrogels. \parencite{PanahiReza2019ProteinHydrogels}
\par
%
Another rheological measurement of interest is strain sweep tests. In these tests the moduli are measured as a function of oscillatory strain, thus information on the strength of the hydrogel network can be obtained. The moduli have a linear dependence on the strain untill the network collapses. At this point $G''$ will become larger than $G'$, as the hydrogel will act more like an elastic fluid, exactly due to the network collapse as the layers in the hydrogel moves with respect to each other. Performing the strain sweep tests thus makes it possible to compare the strength of different hydrogel networks by comparing the linear depandance range and the point of crossover for the moduli. 
\parencite{PanahiReza2019ProteinHydrogels, Seidler2017nativeProteinHydrogels} 
\par
%
\subsection{Protein Hydrogel Particles}
Because of their small size, nanoparticles are a promising tool in drug delivery, as these are able to dissolve  quickly in the bloodstream and reach a specific target. Another advantage of the size is that nanoparticles can be taken up by a cell via endocytosis, resulting in the possibility of drug release directly into the cells. Moreover nanoparticles have a large surface area compared to the volume, enabling rapid drug release, and it has been reported that this is tunable with particle size. Combining these properties with the biocompatebility and biodegradabilty of protein based hydrogels makes protein based hydrogel nanoparticles promising in the field of drug delivery. \parencite{Jacob2018biopolymerBasedNano, Hong2020ProteinBasedNanoparticles}
\par
%
Using nanoparticles as molecule vehicles there are two ways of loading drugs into them: while synthesising the nanoparticles or after synthesis. When loading the drugs while producing the nanoparticles, the drug of interest is merely added to the solutions from which the nanoparticles are produced. On the other hand when loading the drug of interest after nanoparticle production, the nanoparticles are added to highly concentrated drug solutions, and thereby the drugs adsorb to the nanoparticles. It has furthermore been reported that when drugs such as proteins are loaded near the isoelectric point the loading efficiency is proportional to the nanoparticle size. \parencite{Hong2020ProteinBasedNanoparticles}
\par
%
When the loaded drug is released from the nanoparticles there are two major mechanisms: release through diffusion and through particle erosion. The drug release is moreover dependant on the solubility of the drug, as well as the ability of the drug to move in the particle network. \parencite{Hong2020ProteinBasedNanoparticles, Couvreur1993NanoMicropartDelivery}
\par
%
Release of drug through diffusion occurs if the rate of diffusion is faster than the rate of degradation of the nanoparticle in which the drug is stored. When diffusion is the mechanism of drug release only minimal loading of the particles are necessary in order to observe the release of drug. However, if the rate of erosion of the nanoparticle is faster than the rate of diffusion or if the drug is not able to diffuse at all, the drug release is dependant on the rate of the particle degradation. In this case, if performing experiments in vitro, it can be necessary to add enzymes that degrade the particle in order to observe drug release. \parencite{Hong2020ProteinBasedNanoparticles, Couvreur1993NanoMicropartDelivery}



%%Contents%%
    %Hydrogels 
        %definition
        %how it is build up
        %Determining the strength
    %Hydrogel particles
        %What are they and what are they for
        %Metods of drug introduction
        %The mechanisms of drug release
    %Gelatin hydrogel particles
    
    

%General information on hydrogels and their drug delivery properties (Hydrogel droplets for drug delivery - targeted delivery)
%Stability of Hydrogel particles 
%Cross-linking of hydrogels
%Loading cargo in hydrogel particles